\documentclass[12pt,a4paper]{article}
\usepackage[utf8]{inputenc}
\usepackage[T1]{fontenc}
\usepackage[french]{babel}
\usepackage{lmodern}
\usepackage{geometry}
\usepackage{setspace}
\usepackage{hyperref}

\geometry{margin=2.5cm}

\title{Lignes directrices pour la collecte de données\\
et la conformité au droit d'auteur\\
\large{Projet : Modèle de prédiction de suicidabilité chez les poètes}}
\author{José Manuel Rodríguez Caballero}
\date{\today}

\begin{document}

\maketitle
\onehalfspacing

\section{Introduction}

Le présent rapport décrit les principes et les procédures pour la collecte de 
données nécessaires à l'entraînement d'un modèle d'apprentissage automatique 
capable de prédire si un poète est ``suicidaire'' ou non, à partir d'informations 
liées à ses poèmes et à ses caractéristiques démographiques. 

Contrairement à certaines approches qui utilisent des techniques de fouille de texte 
(\textit{text mining}), nous soulignons qu'\textbf{aucun text mining automatisé} 
ne sera mis en \oe uvre. Tous les poèmes seront \textbf{extraits manuellement} de 
livres empruntés en bibliothèque, afin de recueillir les variables nécessaires 
(\textit{e.g.} nombre de mots, mesures de sentiment, etc.). 

Le rapport explique:
\begin{itemize}
    \item Les modalités de collecte des poèmes en accord avec le droit d'auteur.
    \item Les informations à extraire de chaque poème et du poète.
    \item Les aspects \textbf{éthiques} et \textbf{légaux} de la démarche.
\end{itemize}

\section{Objectif du projet}

Notre but est de développer un modèle supervisé (\textit{machine learning}) 
prédictif, qui puisse estimer la probabilité qu'un poète présente un risque de 
suicidabilité, sur la base des informations suivantes:
\begin{itemize}
    \item \textbf{Données démographiques et biographiques} de l'auteur 
    (\textit{genre}, \textit{année de naissance}, \textit{année de décès ou 
    de suicide}, etc.).
    \item \textbf{Caractéristiques agrégées des poèmes} (ex. nombre total 
    de mots, mesures de sentiment, diversité lexicale, etc.).
\end{itemize}

Le projet \textbf{n'exploite pas} le texte intégral des poèmes pour le stockage 
ou la diffusion publique, afin de respecter le droit d'auteur. Seules des mesures 
statistiques et métriques agrégées sont recueillies et enregistrées pour l'analyse.

\section{Collecte de données}

\subsection{Sources des poèmes}

Les poèmes sont consultés dans des livres \textbf{empruntés dans des bibliothèques}. 
Ainsi, chaque poème est extrait \textbf{manuellement} par l'équipe, qui relève 
uniquement des informations structurées:
\begin{itemize}
    \item Titre du poème, titre du livre ou recueil.
    \item Nom de l'auteur, date de naissance et éventuelles autres informations 
    biographiques (si disponibles).
    \item Nombre de mots, nombre de lignes, etc.
    \item Résultats d'analyse sémantique ou de sentiment \textbf{effectués à la main} 
    ou via un logiciel dédié \textit{(localement)}, sans conservation du texte intégral.
\end{itemize}

\subsection{Respect du droit d'auteur}

Dans le cadre de ce projet, nous nous assurons:
\begin{enumerate}
    \item \textbf{D'obtenir légalement l'accès aux livres:} \\
    Les ouvrages sont consultés dans le strict respect des règles de la bibliothèque. 
    Nous n'effectuons aucune reproduction ou numérisation massive 
    qui serait contraire à la législation.

    \item \textbf{De ne pas diffuser le texte intégral:} \\
    Seules des \textbf{extractions agrégées} (comptages, statistiques) 
    seront enregistrées dans notre base de données. 
    Les poèmes complets, même s'ils sont saisissables à la main, 
    ne sont \textbf{pas} inclus dans les données finales.

    \item \textbf{D'utiliser uniquement les mesures nécessaires:} \\
    Pour l'apprentissage automatique, nous ne conservons que 
    les \textbf{indicateurs utiles} (ex. sentiment moyen, 
    proportion de mots négatifs, etc.). Nous ne conservons pas 
    de larges extraits ou de copies du poème susceptibles de violer 
    le droit d'auteur.
\end{enumerate}

\section{Données à recueillir et formalisation}

\subsection{Données sur le poète}

\begin{itemize}
    \item \textbf{Identifiant unique} (numérique ou pseudonyme) permettant 
    de relier plusieurs poèmes au même poète.
    \item \textbf{Nom du poète} (ou anonymisé si nécessaire).
    \item \textbf{Genre} (M, F, Non-binaire, \textit{autre}).
    \item \textbf{Année de naissance}. 
    \item \textbf{Année de suicide} si applicable. 
    \item \textbf{Statut suicidaire} (oui/non) pour la classification.
\end{itemize}

\subsection{Données sur chaque poème}

\begin{itemize}
    \item \textbf{Titre du poème} et \textbf{titre du livre} ou recueil.
    \item \textbf{Nombre de mots} (estimé ou compté).
    \item \textbf{Nombre de lignes} et autres métriques (ex. longueur moyenne 
    de ligne).
    \item \textbf{Scores de sentiment} (ex. entre -1 et +1) 
    ou \textbf{scores d'émotions fines} (ex. colère, joie, tristesse, 
    exprimés en pourcentage).
    \item \textbf{Indicateurs de variabilité} (ex. écart-type du sentiment 
    par ligne, \textit{sentiment slope}).
\end{itemize}

Toutes ces variables sont suffisamment anonymes pour ne pas 
enfreindre le droit d'auteur, dans la mesure où elles ne permettent 
pas de reconstituer directement le texte intégral.

\section{Exploitation des données pour la prédiction}

\subsection{Modèle et pipeline}

\begin{enumerate}
    \item \textbf{Constitution de la base de données}:\\
    Une table reliera chaque poète à ses poèmes (par \textit{poet\_id} 
    et \textit{poem\_id}). 
    Les champs agrégés (sentiment, etc.) seront renseignés.

    \item \textbf{Séparation entraînement/test}:\\
    Nous divisons aléatoirement l'ensemble des poèmes 
    en un échantillon d'entraînement et un échantillon de test.

    \item \textbf{Extraction de caractéristiques}:\\
    Les variables pertinentes (ex. sentiment moyen, score de diversité lexicale, 
    genre du poète, etc.) sont transformées ou encodées (éventuellement normalisées).

    \item \textbf{Entraînement du classifieur}:\\
    Nous pourrons utiliser un algorithme de classification (Random Forest, 
    SVM, réseau de neurones, etc.) afin de distinguer les poètes ``suicidaires'' 
    de ceux qui ne le sont pas.

    \item \textbf{Évaluation et interprétation}:\\
    Les performances seront mesurées via la précision, le rappel, la F-mesure 
    ou encore l'AUC (surface sous la courbe ROC). Nous analyserons également 
    les \textbf{caractéristiques les plus influentes} (importance des variables).
\end{enumerate}

\subsection{Aspects éthiques et limitations}

\begin{itemize}
    \item \textbf{Sujet sensible}: 
    L'étude de la suicidabilité peut engendrer des interprétations délicates. 
    Le modèle n'a pas vocation à poser un diagnostic médical.

    \item \textbf{Variabilité historique et culturelle}: 
    Les poètes sélectionnés peuvent varier selon l'époque, la culture, 
    et la langue. Les conclusions ne sont pas nécessairement généralisables 
    à l'ensemble des populations.

    \item \textbf{Respect de la vie privée et de la mémoire}: 
    Les données ne doivent pas diffamer ni stigmatiser des auteurs spécifiques. 
    L'anonymisation peut s'avérer nécessaire si les poètes ou leurs ayants droit 
    le demandent.
\end{itemize}

\section{Conclusion}

Ce document présente les lignes directrices pour \textbf{collecter et traiter} 
des informations extraites \textbf{manuellement} de poèmes empruntés en bibliothèque, 
sans enfreindre le droit d'auteur. Les grandes idées sont:
\begin{itemize}
    \item \textbf{Limiter la collecte} aux variables agrégées et indispensables 
    au modèle prédictif.
    \item \textbf{Ne pas reproduire} les oeuvres intégrales pour éviter toute 
    violation de la propriété intellectuelle.
    \item \textbf{Privilégier la prudence} sur le plan éthique et légal, 
    compte tenu du sujet sensible (le risque suicidaire).
\end{itemize}

Les données, une fois rassemblées, alimenteront un modèle d'apprentissage 
automatique conçu pour \textit{prédire} la suicidabilité d'un poète sur 
la base de caractéristiques linguistiques et démographiques. Les résultats 
devront être interprétés avec \textbf{caution} et ne sauraient se substituer 
à un diagnostic médical.

\bigskip
\textbf{Référence Principale :}
\begin{itemize}
    \item \textbf{Stirman, Shannon Wiltsey, and James W. Pennebaker.} 
    ``Word use in the poetry of suicidal and nonsuicidal poets.'' 
    \textit{Psychosomatic medicine} 63, no. 4 (2001): 517-522.
\end{itemize}

\end{document}

