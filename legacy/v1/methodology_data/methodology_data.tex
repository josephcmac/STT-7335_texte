\documentclass[12pt,a4paper]{article}

% Encodage et langues
\usepackage[utf8]{inputenc}    
\usepackage[T1]{fontenc}       


% Mise en page
\usepackage{geometry}
\geometry{margin=2.5cm}

% Interligne
\usepackage{setspace}

% Liens hypertextes
\usepackage{hyperref}

% Pour inclure des URL simples
\usepackage{url}

\title{Description du Jeu de Données, Tâche de Prévision \\
	et Lignes Directrices de Collecte \\
	\large{Projet : Modèle de prédiction de suicidabilité chez les poètes}}
\author{José Manuel Rodríguez Caballero}
\date{\today}

\begin{document}
	
	\maketitle
	\onehalfspacing
	
	\begin{abstract}
		Ce document présente les jeux de données (public et privé) relatifs aux poètes 
		et à leur niveau de ``suicidabilité'', ainsi que les modalités de collecte des 
		poèmes dans le respect du droit d'auteur. Nous décrivons également la tâche de 
		prévision, qui s'appuie sur un ensemble de \textbf{caractéristiques numériques} 
		extraites des textes (scores d'émotions, lexiques sentimentaux, etc.). Pour 
		certains champs comme l’orientation sexuelle (\textit{heterosexual}), des valeurs 
		sont manquantes et feront l’objet d’une imputation afin de ne pas exclure 
		d’observations importantes dans le modèle prédictif.
	\end{abstract}
	
	%--------------------------------------------------------------------
	\section{Introduction}
	\label{sec:intro}
	
	Le présent document a pour but de décrire :
	\begin{itemize}
		\item \textbf{Les jeux de données} utilisés, avec deux versions : privé et public.
		\item \textbf{La tâche de prévision} : modéliser la probabilité qu’un poète 
		présente un risque de suicidabilité, en se basant sur ses caractéristiques 
		démographiques et linguistiques.
		\item \textbf{La collecte} et le \textbf{traitement} des textes poétiques 
		(extractions manuelles, création de variables agrégées) dans le respect 
		de la législation sur le droit d'auteur.
	\end{itemize}
	
	Contrairement à certaines approches de fouille de texte (\textit{text mining}) 
	entièrement automatisées, nous veillons à \textbf{ne pas} reproduire l’intégralité 
	des poèmes, qui sont protégés par le droit d’auteur. Les données textuelles brutes 
	ne figurent \textbf{que} dans le jeu de données \textit{privé}, lequel n'est pas 
	diffusé. Le jeu \textit{public}, quant à lui, ne contient que des mesures numériques 
	extraites (sentiment, longueur moyenne des mots, etc.).
	
	%--------------------------------------------------------------------
	\section{Description du Jeu de Données}
	\label{sec:dataset_description}
	
	\subsection{Jeu de données privé (\texttt{case-control-long.csv})}
	Ce fichier contient :
	\begin{itemize}
		\item \textbf{Données démographiques} : sexe, orientation sexuelle (hétérosexuel 
		ou non), date de naissance, date de décès, pays d’origine, etc.
		\item \textbf{Textes de poèmes} : le texte intégral pour chaque poème, 
		permettant des analyses approfondies. 
		\item \textbf{Étiquettes de suicidabilité} : variable binaire indiquant si 
		le poète est considéré ``suicidaire'' (TRUE) ou non (FALSE).
	\end{itemize}
	
	Ce jeu de données est \textbf{strictement confidentiel}, car il inclut 
	les poèmes complets, protégés par le droit d’auteur. Il n’est utilisé 
	\textit{que} pour le développement du modèle au sein du groupe de recherche 
	et sous la supervision de l’enseignant.
	
	\subsection{Jeu de données public (\texttt{case-control-clean.csv})}
	Afin de respecter le droit d’auteur et de pouvoir partager une partie des informations, 
	nous avons créé une version ``nettoyée'' :
	\begin{itemize}
		\item \textbf{Colonnes démographiques} : identiques (nom, sexe, dates, etc.), 
		à l’exception des champs sensibles qui sont anonymisés ou supprimés 
		si nécessaire.
		\item \textbf{Caractéristiques textuelles agrégées} : 
		\begin{itemize}
			\item \textbf{Scores émotionnels} (tristesse, joie, confusion, colère, 
			peur, surprise, dégoût, espoir), obtenus grâce à un \textit{dictionary-based approach}.
			\item \textbf{Scores de sentiment} (\texttt{afinn}, \texttt{bing}, \texttt{nrc\_ratio}).
			\item \textbf{Score lexical suicidaire} (proportion de mots liés à l’idée 
			de mort, de fin, etc.).
			\item \textbf{Longueur moyenne} des mots (\texttt{mean\_length}).
		\end{itemize}
		\item \textbf{Statut suicidaire} (TRUE/FALSE).
	\end{itemize}
	
	Ici, \textbf{aucun} texte intégral de poème n’est conservé. Les valeurs 
	manquantes (par exemple l’orientation sexuelle pour certains poètes) 
	seront traitées via des \textbf{méthodes d’imputation}, afin d’éviter 
	des biais importants dans l’analyse.
	
	%--------------------------------------------------------------------
	\section{Collecte des Données et Respect du Droit d'Auteur}
	\label{sec:data_collection}
	
	\subsection{Sources des poèmes}
	\begin{itemize}
		\item \textbf{Recueils en bibliothèque} : consultation légale et manuelle 
		(lecture physique des livres).
		\item \textbf{Sites web autorisés} tels que \url{https://poetryarchive.org/} 
		ou \url{https://www.poetryfoundation.org/}, pour récupérer \textbf{quelques} 
		poèmes nécessaires à l’étude.
	\end{itemize}
	
	\subsection{Procédure de collecte}
	\begin{enumerate}
		\item \textbf{Extraction manuelle} : les poèmes sont copiés \textit{à la main} 
		afin d’en extraire les informations utiles (nombre de mots, sentiment, etc.).
		\item \textbf{Pas de reproduction intégrale} : seule la base privée contient 
		les textes complets, et \textbf{n’est pas} partagée publiquement.
		\item \textbf{Nettoyage et agrégation} : les mesures de sentiment, d’émotions, 
		de longueur, etc., sont calculées localement, puis conservées comme 
		\textit{features numériques} dans le jeu \texttt{case-control-clean.csv}.
	\end{enumerate}
	
	%--------------------------------------------------------------------
	\section{Extraction des Caractéristiques Textuelles}
	\label{sec:text_features}
	
	\subsection{Emotion et Sentiment Lexicons}
	Une \textbf{approche par dictionnaires} (\textit{dictionary-based approach}) 
	est mise en place pour calculer plusieurs scores :
	\begin{itemize}
		\item \textbf{Émotions de base} : tristesse, joie, confusion, colère, peur, 
		surprise, dégoût, espoir. Chaque mot du poème est comparé à des lexiques 
		d’émotions, puis \textbf{pondéré} selon la position de mots de négation 
		ou d’intensification.
		\item \textbf{Lexiques AFINN, Bing, et NRC} : 
		\begin{itemize}
			\item \textbf{AFINN} : score de -5 à +5 pour chaque mot (colère, joie, etc.).
			\item \textbf{Bing} : catégorisation binaire (positif / négatif).
			\item \textbf{NRC ratio} : ratio (positif - négatif) / total.
		\end{itemize}
		\item \textbf{Score lexical suicidaire} : proportion de mots associés à la mort 
		ou au désespoir (\texttt{despair, suicide, death, ...}).
	\end{itemize}
	
	\subsection{Longueur moyenne des mots}
	Pour chaque poème, on calcule également la taille moyenne des tokens. 
	Cette variable peut être indicative du style poétique ou du registre de langue.
	
	\subsection{Gestion des données manquantes}
	Pour la variable \texttt{heterosexual} (orientation sexuelle), certaines valeurs 
	sont absentes. Nous appliquerons des \textbf{méthodes d’imputation statistique} 
	(e.g. \textit{multiple imputation by chained equations} - MICE) afin de 
	\textbf{ne pas} exclure ces poètes des analyses. Dans la mesure où l’orientation 
	peut être pertinente pour étudier d’éventuels biais socioculturels, 
	il importe de compléter ces données prudemment, sans introduire d’erreurs majeures.
	
	%--------------------------------------------------------------------
	\section{Tâche de Prévision}
	\label{sec:forecast_task}
	
	\subsection{Objectif principal}
	L’objectif est de \textbf{prédire la suicidabilité d’un poète} (variable binaire 
	\texttt{suicidal = TRUE/FALSE}) à partir des informations suivantes :
	\begin{itemize}
		\item \textbf{Données démographiques} (sexe, orientation — éventuellement imputée, 
		date de naissance, pays, etc.).
		\item \textbf{Caractéristiques textuelles} (scores d’émotions, mesures de sentiment, 
		longueur moyenne, etc.).
	\end{itemize}
	
	\subsection{Modélisation}
	\begin{itemize}
		\item \textbf{Régression logistique hiérarchique} à trois niveaux : 
		\begin{enumerate}
			\item \textbf{Pair cas-témoin} : un poète suicidaire (cas) et 
			un poète non suicidaire (témoin), appariés selon la période, 
			le genre, etc.
			\item \textbf{Niveau poète} : agrégation de plusieurs poèmes par auteur.
			\item \textbf{Niveau poème} : chaque texte constitue une sous-unité 
			d’observation (scores de sentiment, etc.).
		\end{enumerate}
		\item \textbf{Autres approches supervisées} : Random Forest, SVM, réseaux de neurones, etc., 
		pour comparer les performances.
		\item \textbf{Gestion des données manquantes} : intégration d’un schéma 
		d’imputation (MICE ou équivalent) pour la variable \texttt{heterosexual}, 
		afin de réduire les biais de sélection.
	\end{itemize}
	
	\subsection{Pipeline}
	\begin{enumerate}
		\item \textbf{Prétraitement} : imputation des données manquantes, normalisation 
		des variables si nécessaire.
		\item \textbf{Séparation Entraînement/Test} : division aléatoire du jeu public (ou privé) 
		en ensembles de modélisation et de validation.
		\item \textbf{Entraînement du modèle} : régression logistique hiérarchique 
		(ou autre algorithme).
		\item \textbf{Évaluation} : précision, rappel, F-mesure, AUC (ROC). 
		Interprétation des coefficients ou importance des variables.
	\end{enumerate}
	
	%--------------------------------------------------------------------
	\section{Aspects Éthiques et Limitations}
	\label{sec:ethics}
	
	\begin{itemize}
		\item \textbf{Sujet sensible} : La prédiction de la suicidabilité chez un individu 
		(même un poète décédé) doit être manipulée avec prudence ; la santé mentale 
		n’est pas réductible à de simples variables quantitatives.
		\item \textbf{Risques de mauvaise interprétation} : Les utilisateurs pourraient 
		faire des généralisations hâtives ou transformer ces analyses en jugement 
		de valeur sur la vie d’un auteur.
		\item \textbf{Données manquantes} : L’imputation, bien que utile, reste imparfaite 
		et peut introduire des incertitudes. Les résultats doivent être interprétés 
		en tenant compte de ces biais potentiels.
		\item \textbf{Droit d’auteur} : La non-diffusion du texte intégral demeure impérative 
		pour respecter la législation en vigueur. Seules les mesures agrégées sont publiques.
	\end{itemize}
	
	%--------------------------------------------------------------------
	\section{Conclusion}
	\label{sec:conclusion}
	
	Dans ce document, nous avons présenté :
	\begin{itemize}
		\item Les deux versions du jeu de données, privé (avec textes intégraux) 
		et public (avec features agrégées).
		\item La \textbf{méthodologie de collecte}, fondée sur une extraction manuelle 
		et minutieuse, respectant le droit d'auteur.
		\item Les \textbf{caractéristiques textuelles} dérivées (émotions, sentiment, longueur, etc.).
		\item L’importance de l’\textbf{imputation des valeurs manquantes} (particulièrement 
		pour l’orientation sexuelle) afin de garder un maximum de poètes dans l’analyse.
		\item Le \textbf{modèle de prévision} envisagé (régression logistique hiérarchique) 
		et les possibles alternatives (Random Forest, SVM, etc.).
	\end{itemize}
	
	Cette approche vise à comprendre si la production poétique peut renseigner 
	sur un risque de suicidabilité, tout en \textbf{préservant la confidentialité} 
	des oeuvres et le respect dû aux auteurs. Les résultats, quel que soit le 
	taux de réussite du modèle, doivent être interprétés avec \textbf{prudence} 
	et ne sauraient se substituer à un diagnostic médical ou psychiatrique.
	
	\bigskip
	\noindent
	\textbf{Référence Principale :} \\
	\textit{Stirman, Shannon Wiltsey, and James W. Pennebaker. 
		"Word use in the poetry of suicidal and nonsuicidal poets." 
		Psychosomatic medicine 63, no. 4 (2001): 517-522.}
	 $\newline$
	 
	\noindent
    \textbf{Projet sur GitHub :} $\newline$
     
      \quad Texte : \url{https://github.com/josephcmac/STT-7335_texte} \\
     
     \quad Code : \url{https://github.com/josephcmac/STT-7335_code}
	
\end{document}
