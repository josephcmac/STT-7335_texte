\section{Introduction}
\label{sec:introduction}

La présente étude vise à déterminer dans quelle mesure on peut distinguer 
les poètes suicidaires des poètes non suicidaires, à partir de différentes 
variables démographiques (par exemple le sexe ou l’orientation sexuelle) 
et d’indicateurs émotionnels (colère, joie, confiance, etc.). Dans ce cadre, 
nous avons constitué un jeu de données comprenant 16 poètes, 3 poèmes par poète et 3443 vers, chacun associé 
à plusieurs observations des scores émotionnels. Des techniques 
statistiques variées — allant du rééchantillonnage \emph{bootstrap} à la 
réduction de dimension par \emph{Analyse en Composantes Principales} (ACP), 
en passant par des algorithmes de classification (régression logistique, 
XGBoost, mélanges gaussiens) — ont été mobilisées pour explorer la robustesse 
de la distinction suicidaire \emph{vs.} non suicidaire. L’objectif global est 
d’évaluer la fiabilité de ces approches, tant sur le plan méthodologique 
(gestion de données manquantes, validation \emph{leave-one-poet-out}) que 
sur le plan pratique (comparaison des performances et limites inhérentes 
au faible effectif).