\section{Comparaison des trois modèles de classification}
\label{sec:comparaison-modeles}

Dans cette section, nous comparons trois techniques de classification afin de prédire si un
poète appartient à la classe \og\, suicidaire \fg{} ou à la classe \og\, non suicidaire \fg. Les
méthodes considérées sont :

\begin{itemize}
	\item \textbf{XGBoost}, qui repose sur un algorithme de \emph{gradient boosting},
	\item \textbf{Régression logistique}, associée à un modèle linéaire généralisé (lien logit),
	\item \textbf{Modèle gaussien}, où chaque classe suit une loi normale multivariée
	(analyse discriminante).
\end{itemize}

\subsection{Procédure d’évaluation}

Un ensemble de données, constitué de poètes \og\, suicidaires \fg{} et \og\, non suicidaires \fg, 
est subdivisé en un échantillon d’apprentissage et un échantillon de test. Pour renforcer 
la robustesse des estimations, l’échantillon d’apprentissage est rééchantillonné de 
multiples fois (\emph{bootstrap}), et chacun des trois algorithmes de classification 
est ajusté sur ces rééchantillonnages. Les poètes conservés hors de l’apprentissage 
servent alors à tester la capacité de généralisation de chaque méthode.  
En répétant ce processus plusieurs fois, on agrège l’ensemble des prédictions afin 
d’obtenir une \emph{matrice de confusion globale} pour chaque modèle, accompagnée 
des principales mesures de performance.

Pour avoir une \textbf{vue d’ensemble} des résultats, nous proposons de présenter séparément 
les \textbf{matrices de confusion} (une par modèle) et \textbf{les mesures de performance} 
(les indicateurs usuels) dans deux séries de tableaux.  

\begin{table}[H]
	\centering
	\caption{Matrice de confusion pour \textbf{XGBoost}.}
	\label{tab:cm-xgboost-separee}
	\begin{tabular}{cc|cc}
		\toprule
		& & \multicolumn{2}{c}{\textbf{Prédit}} \\
		& & \textbf{FAUX} & \textbf{VRAI} \\
		\midrule
		\textbf{Observé} 
		& \textbf{FAUX} & 30 & 17 \\
		& \textbf{VRAI} & 10 & 43 \\
		\bottomrule
	\end{tabular}
\end{table}

\begin{table}[H]
	\centering
	\caption{Matrice de confusion pour la \textbf{Régression Logistique}.}
	\label{tab:cm-glm-separee}
	\begin{tabular}{cc|cc}
		\toprule
		& & \multicolumn{2}{c}{\textbf{Prédit}} \\
		& & \textbf{FAUX} & \textbf{VRAI} \\
		\midrule
		\textbf{Observé}
		& \textbf{FAUX} & 36 & 17 \\
		& \textbf{VRAI} & 6  & 41 \\
		\bottomrule
	\end{tabular}
\end{table}

\begin{table}[H]
	\centering
	\caption{Matrice de confusion pour le \textbf{Modèle Gaussien}.}
	\label{tab:cm-gaussien-separee}
	\begin{tabular}{cc|cc}
		\toprule
		& & \multicolumn{2}{c}{\textbf{Prédit}} \\
		& & \textbf{FAUX} & \textbf{VRAI} \\
		\midrule
		\textbf{Observé}
		& \textbf{FAUX} & 24 & 21 \\
		& \textbf{VRAI} & 0  & 55 \\
		\bottomrule
	\end{tabular}
\end{table}

\noindent
Les \textbf{Tables~\ref{tab:cm-xgboost-separee}, \ref{tab:cm-glm-separee} et \ref{tab:cm-gaussien-separee}}
montrent le nombre d’occurrences pour chacune des quatre cellules (vrais négatifs, faux positifs, 
faux négatifs, vrais positifs) en fonction du modèle considéré. Rappelons que \textbf{FAUX} et \textbf{VRAI} 
désignent respectivement \og non suicidaire \fg{} et \og suicidaire \fg, et que la \og classe positive \fg{} 
est donc \textbf{VRAI}.

\vspace{1em}

\begin{table}[H]
	\centering
	\caption{Principales mesures de performance pour chacun des trois modèles.}
	\label{tab:comparaison_mesures}
	\begin{tabular}{p{4.2cm}ccc}
		\toprule
		\textbf{Indicateur} 
		& \textbf{XGBoost} 
		& \textbf{Logistique}
		& \textbf{Gaussien} \\
		\midrule
		Exactitude (\emph{Accuracy})              
		& 0,73          
		& 0,77          
		& 0,79          \\
		IC à 95\,\% (Exactitude)            
		& [0,632 -- 0,8139]
		& [0,6751 -- 0,8483]
		& [0,6971 -- 0,8651] \\
		Sensibilité (\emph{Sensitivity})          
		& 0,7167       
		& 0,7069        
		& 0,7237        \\
		Spécificité (\emph{Specificity})          
		& 0,7500        
		& 0,8571        
		& 1,0000        \\
		Valeur préd. positive (\emph{VPP})        
		& 0,8113        
		& 0,8723        
		& 1,0000        \\
		Valeur préd. négative (\emph{VPN})        
		& 0,6383        
		& 0,6792        
		& 0,5333        \\
		Kappa                                      
		& 0,4534        
		& 0,5444        
		& 0,557         \\
		p-value (Test de McNemar)                  
		& 0,2482        
		& 0,03706       
		& 1,275$\times 10^{-5}$ \\
		\bottomrule
	\end{tabular}
	
	\vspace{1ex}
	\footnotesize
	\textit{Remarque :} la \textbf{classe positive} est \textbf{VRAI} (suicidaire).
\end{table}

