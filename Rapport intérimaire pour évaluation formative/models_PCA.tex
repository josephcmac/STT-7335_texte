\section{Modélisation par mélanges gaussiens et ACP}
\label{sec:mgaussiens-pca}

L’objectif de ce chapitre est d’évaluer la capacité d’un \textbf{modèle de mélanges gaussiens} (\texttt{MclustDA}) à distinguer les poètes \textbf{suicidaires} (\og\, VRAI \fg) des \textbf{non suicidaires} (\og\, FAUX \fg), en s’appuyant sur différentes variables émotionnelles et démographiques. Pour renforcer la robustesse du modèle, on applique notamment un \textbf{rééchantillonnage bootstrap} suivi d’une \textbf{Analyse en Composantes Principales (ACP)}. Les prédictions sont finalement validées à l’aide d’une procédure \og\, \emph{leave-one-poet-out} \fg.

\subsection{Protocole général}

\begin{enumerate}
	\item \textbf{Lecture et préparation des données.}  
	Les observations portent sur 16 poètes, pour lesquels on dispose de :
	\begin{itemize}
		\item leurs \emph{caractéristiques émotionnelles} (par exemple \texttt{anticipation}, \texttt{colère}, \texttt{joie}, \texttt{positive}, etc.), 
		\item leur statut \og\, suicidaire \fg{} ou \og\, non suicidaire \fg, encodé en \textbf{VRAI/FAUX}.
	\end{itemize}
	Après la lecture du fichier \texttt{clean\_data\_2.csv}, les colonnes appropriées sont sélectionnées et converties aux formats voulus (facteurs, booléens, etc.).
	
	\item \textbf{Création d’échantillons bootstrap.}  
	Pour chaque itération, un nouveau \emph{jeu d’entraînement} est généré \emph{avec remise} (10\,000 rééchantillonnages), et la moyenne des variables numériques est calculée séparément pour les poètes suicidaires (VRAI) et non suicidaires (FAUX). Cette étape vise à lisser la variabilité, surtout lorsque l’échantillon est restreint.
	
	\item \textbf{Réduction de dimension par ACP.}  
	Sur chaque jeu d’entraînement bootstrapé, on applique une \textbf{analyse en composantes principales}. Les variables émotionnelles, centrées et réduites, sont projetées dans un nouvel espace de dimensions potentiellement plus faibles. Cela limite les effets de la colinéarité et facilite l’ajustement du modèle gaussien.
	
	\item \textbf{Ajustement du modèle \texttt{MclustDA}.}  
	Les scores sur les composantes principales (ACP) et la variable cible \texttt{suicidal} (VRAI/FAUX) servent de base pour estimer un \textbf{modèle de mélanges gaussiens discriminants}. Pour un nouvel individu, le modèle alloue la classe (VRAI ou FAUX) qui maximise la probabilité a posteriori.
	
	\item \textbf{Validation \og\, leave-one-poet-out\fg.}  
	À chaque itération, on \emph{exclut} un poète afin de constituer l’échantillon de test. Le reste des données est utilisé pour :
	\begin{itemize}
		\item générer le bootstrap et ajuster le modèle (\texttt{MclustDA}) après ACP,
		\item projeter le poète exclu dans l’espace des composantes principales, afin de prédire son statut suicidaire (VRAI/FAUX).
	\end{itemize}
	Cette procédure est répétée pour les 16 poètes, chacun étant tour à tour l’exclu.
	
\end{enumerate}

\subsection{Résultats de la validation leave-one-poet-out}

Le Tableau~\ref{tab:res-by-poet} détaille, pour chaque poète, la classe \textbf{observée} (VRAI = suicidaire, FAUX = non suicidaire) et la classe \textbf{prédite} (VRAI/FAUX) par le modèle après application du protocole. Les poètes dont l’observation et la prédiction ne coïncident pas apparaissent en gras.

\begin{table}[H]
	\centering
	\caption{Comparaison entre la classe prédite et la classe observée pour chaque poète (validation leave-one-poet-out).}
	\label{tab:res-by-poet}
	\begin{tabular}{lcc}
		\toprule
		\textbf{Poète} & \textbf{Observé (suicidal)} & \textbf{Prédit (suicidal)} \\
		\midrule
		Sylvia Plath            & VRAI  & VRAI  \\
		Denise Levertov         & FAUX & FAUX \\
		Anne Sexton             & VRAI  & VRAI  \\
		\textbf{Adrienne Rich}           & FAUX & VRAI  \\
		Randall Jarrell         & VRAI  & VRAI  \\
		Robert Lowell           & FAUX & FAUX \\
		John Berryman           & VRAI  & VRAI  \\
		Lawrence Ferlinghetti   & FAUX & FAUX \\
		Hart Crane              & VRAI  & VRAI  \\
		William Carlos Williams & FAUX & FAUX \\
		Sara Teasdale           & VRAI  & VRAI  \\
		\textbf{Edna St. Vincent Millay} & FAUX & VRAI  \\
		Charlotte Mew           & VRAI  & VRAI  \\
		Edith Sitwell           & FAUX & FAUX \\
		John Davidson           & VRAI  & VRAI  \\
		\textbf{Alfred Edward Housman}  & FAUX & VRAI  \\
		\bottomrule
	\end{tabular}
\end{table}


\subsection{Matrice de confusion et indicateurs de performance}

En agrégeant l’ensemble des prédictions issues du tableau précédent, on obtient la \textbf{matrice de confusion globale} (Tableau~\ref{tab:conf-matrix}) :

\begin{table}[H]
	\centering
	\caption{Matrice de confusion globale issue du protocole \emph{leave-one-poet-out}.}
	\label{tab:conf-matrix}
	\begin{tabular}{lcc}
		\toprule
		& \textbf{Observé : Non-suicidaire} & \textbf{Observé : Suicidaire} \\
		\midrule
		\textbf{Prédit : Non-suicidaire} & 5 & 0 \\
		\textbf{Prédit : Suicidaire}     & 3 & 8 \\
		\bottomrule
	\end{tabular}
\end{table}

La Table~\ref{tab:conf-stats} regroupe plusieurs indicateurs dérivés de cette matrice :

\begin{table}[H]
	\centering
	\caption{Statistiques de performance du modèle \texttt{MclustDA} (validation leave-one-poet-out).}
	\label{tab:conf-stats}
	\begin{tabular}{lc}
		\toprule
		\textbf{Mesure} & \textbf{Valeur estimée} \\
		\midrule
		Exactitude (Accuracy)            & \(81.25\%\) \\
		Intervalle de confiance (95\,\%)  & \([54.35\%,\,95.95\%]\) \\
		Kappa                             & \(0.625\) \\
		Sensibilité (Recall)             & \(1.00\) \\
		Spécificité                       & \(0.625\) \\
		Exactitude équilibrée (Balanced Accuracy) & \(0.8125\) \\
		\bottomrule
	\end{tabular}
\end{table}

\noindent
\textbf{Faits saillants :}
\begin{itemize}
	\item \textbf{Sensibilité = 1,00} : le modèle détecte tous les poètes suicidaires (aucun \emph{faux négatif}).  
	\item \textbf{Spécificité = 0,625} : près d’un tiers des poètes non suicidaires sont étiquetés à tort comme suicidaires (\emph{faux positifs}).  
	\item \textbf{Exactitude globale} (Accuracy) : environ \(81\,\%\), avec un intervalle de confiance assez large, reflétant la petite taille de l’échantillon.  
	\item \textbf{Kappa = 0,625} : indique une concordance modérée entre la classification prédite et la classification observée, au-delà du pur hasard.
\end{itemize}

